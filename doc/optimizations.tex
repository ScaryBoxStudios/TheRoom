\section{Optimizations}

\subsection{File Loading}
When loading a data resource, the biggest bottleneck is in the fetch from a persistent storage medium.
Doing multiple IO operations on a hard disk for example, can heavily increase the load time of a file
due to their mix with other IO operations from the rest of the Operating System\footnote{Programs do
not have exclusive ownership of the hard disk in a multitasking environment}. With this in mind, resources
from Hard Drive or other persistent storages are best loaded with the fewest IO operations that can be done.
In our case, files are being read in a single shot.

\subsection{Rendering State Changes Minimization}
Changing often the driver state (through OpenGL calls for example), can also steal precious time from the program.
This occurs because a possible driver call may change the state in our graphics card hardware or not depending
on the action we took.\footnote{Although, modern drivers often try to defer as much state change as possible
till the next draw call} Therefore, redundant graphics API calls are avoided and similar ones are batched
as much as possible.

\subsection{Deferred Rendering}
With the traditional Forward Rendering pipeline, each vertex passes through the Gpu processing pipeline
regardless if it is visible or not in the end. That means that lots of fragments are reaching the final
shading stage without any actual effect. In addition, this approach does not scale well with more lights
as the more lights we have, the more unnecessary shading operations we get. To avoid this rendering overhead
we need a way to be able to shade only the final visible fragments. This is achieved through a ``Deferred
Rendering'' pipeline that is composed in 2 main passes:

\begin{enumerate}
\item Geometry Pass
\item Lighting Pass
\end{enumerate}

\subsubsection{Geometry Pass}

\begin{figure}[ht]
    \centering
    \includegraphics[scale=0.3, clip=true]{./image/gbuffer.png}
    \caption{Position, Normal and BaseColor GBuffer textures}
\label{fig:gbuf}
\end{figure}

In the Geometry Pass we render each object once. Base color, metalness, roughness, reflectivity, and world space normals
are rendered into framebuffer targets (textures) for each final fragment. This logical grouping of textures is named
a GBuffer (Geometry Buffer). The GBuffer is set as a Multiple Render Target output in the Geometry Pass, so in the end
of the fragment shading stage all the above mentioned data are saved to their respective texture in the GBuffer which,
in turn is bound and used when shading the given fragment in the Lighting Pass. The visualized contents of some GBuffer
textures can be seen in figure~\ref{fig:gbuf}.

\subsubsection{Lighting Pass}

The lighting pass consists of multiple individual light passes one per direct light plus one for the
environmental lighting. For each one sub-pass we compute the lighting for each screen space fragment using
the data from the GBuffer and the bound ShadowMaps. Starting with an empty accumulation texture buffer (black)
we add the final light contribution from every pass. For directional lights and environmental lighting a full screen
quad render is performed, while for point lights only fragments of the area they affect are being rendered.

\begin{figure}[h]
    \centering
    \includegraphics[scale=0.18, clip=true]{./image/lightpasses.png}
    \caption{Directional light, Environmental light and Accumulated light}
\label{fig:lpasses}
\end{figure}

\subsection{Bounding Spheres Optimization}

When making the point light passes it is desirable that we shade only the visible fragments that reside inside the light
sphere's radius. The first attempt on this would be the rendering of a sphere around the point light position instead
of a screen quad pass. The problems with this simplistic approach are imminent, first of no actual bounding is occuring
as the projection of the rendered sphere is a circle and objects out of it (in front or behind it) are also lit.
Secondly as soon as the viewer enters the bounding volume the point light disappears due to the back face culling
being enabled. Disabling back face culling would not be a viable option as this leads to increased light when outside
the sphere (because we render both faces) and inverse of the original effect if only front face culling is enabled (light
only when inside the volume).
To bypass these problems a smart trick using the stencil buffer can be used. It can be separated in two main parts:
\begin{enumerate}
    \item Mark affected fragments in the Stencil Buffer
    \item Render sphere with Stencil Test enabled
\end{enumerate}
In order to mark the affected fragments in the Stencil Buffer we rely on the fact that when we look on the bounding
sphere from the camera point of view:
\begin{enumerate}
    \item Both bounding sphere's front and back face polygons are behind an object that is in front it
    \item Both bounding sphere's front and back face polygons are in front of an object that is behind it
    \item The front face polygons are infront but the back face polygons are behind of an object that is inside the bounding sphere
\end{enumerate}
So in order to use this with the Stencil Buffer we take the following steps:
\begin{enumerate}
    \item Disable writing into the depth buffer, making it read-only
    \item Disable back face culling, in order to process all the polygons of the sphere
    \item Set the stencil test to always succeed (What we really care is the operation)
    \item Configure the stencil operation for back facing polygons to increment the value in the stencil buffer
        when the depth test fails but keep it unchanged when either depth test or stencil test succeeds
    \item Configure the stencil operation for front facing polygons to decrement the value in the stencil buffer
        when the depth test fails but keep it unchained when either depth test or stencil test succeeds
    \item Render the light sphere using null shaders and disabled output to affect only stencil buffer
\end{enumerate}
This way when an object is outside the bounding volume the stencil buffer is balanced out by the decrement of the
front face polygons and the increment of the back face polygons of the bounding sphere. The result is a stencil buffer
with non zero parts of the object fragments that are affected by the light. Using this the point light can now be rendered
with the stencil test enabled and passing when the stencil value is not equal to zero. Last but not least,
front face culling must be enabled before making the point light pass; this must be done because the viewer may be inside
the light volume and if back face culling is used as normally, light will not be visible until viewer exit its volume.
