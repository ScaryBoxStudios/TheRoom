\section{Cascaded Shadow Maps}

In order to implement shadows in the scenes the use of some Shadow Mapping technique was required.
The simple shadow mapping algorithm would not be sufficient for large space scenes. Using just
a single shadowmap leads to difficulties in adjusting it to avoid surface acne and aliasing
problems. Additionally, this would require a very high and impractical resolution in order capture both large
space scenes and being able to show the desirable level of detail. Cascaded Shadow Maps approach
helps bypass these problems by providing higher resolution of the depth texture near the viewer lower
resolution far away. This is achieved by splitting the camera view frustum and creating a separate depth-map
for each partition (see Figures~\ref{fig:csmsplits},\ref{fig:csmbufs}).

\begin{figure}[h]
    \centering
    \includegraphics[scale=0.25, clip=true]{./image/csmsplits.png}
    \caption{Visualized frustum splits}
\label{fig:csmsplits}
\end{figure}

The algorithm is composed of two main parts:
\begin{enumerate}
\item For every frustum split render the scene depth from the light point of view (POV)
\item While rendering the scene from the camera POV, pick and use the matching shadow map for the current
    fragment's Z value
\end{enumerate}

For the creation of each depth map, the split planes, view and projection matrices for each split
must be first calculated. The procedure for each split is:

\begin{enumerate}
\item Calculate near and far split distances
\item Create the projection for the split using previously calculated near and far values
\item Extract the split frustum points by multiplying the inverse projection-view matrix
    with each of the vertices of an NDC cube (note that the prespective division must
    be performed manually)
\item Find the centroid of the previously calculated split frustum
\item Create the view matrix with the eye being the light position looking towards the centroid
\item Transform the split vertices to the light view space
\item Find the frustum bounding box in view space
\item Create an orthogonal projection matrix with the corners of the previously calculated bounding box
\item Save calculated split projection, view matrices and near, far planes for later use
\end{enumerate}

Next, the scene needs to be rendered for each split for the purpose of creating the shadow maps.
In each split render, the vertices are transformed by their respective projection and view matrices
that have been already calculated. Note that the final stored depths must be in linear space.
An intermediate geometry shader with invocations number set to the split number can also be used
to avoid multiple draws. Before rendering the shadow maps it is recommended to switch to front
face culling to avoid peter-panning effects~\footnote{Petter Panning effect is when objects with missing shadows
appear to be detached from and to float above the surface, just like Peter Pan.}.

When using the shadow maps in the shading phase, all that needs to be done is choose the shadow
map layer to use according to the depth of the current vertex in view space. After that
the fragment must be transformed from world to light space and check if it is in shadow like
the normal shadowmapping algorithm. Optional PCF can be applied after that.

\begin{figure}[b]
    \centering
    \includegraphics[scale=0.3, clip=true]{./image/csmbufs.png}
    \caption{Depth maps for each frustum split}
\label{fig:csmbufs}
\end{figure}

